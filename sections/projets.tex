{\large Suivi du cours CS231n 2017, Stanford university, \href{https://github.com/ThomasRanvier/stanford-cs231n}{(github)}}
\\
{\small Le cours est de niveau Master, son libell\'e complet est ``Convolutional Neural Networks for Visual Recognition''.
Il est compos\'e de cours magistraux disponibles en rediffusion, de cours \'ecrits disponibles sur leur site ainsi que de 3 assignments sous forme de notebooks Jupyter permettant d'impl\'ementer les connaissances acquises tout en v\'erifiant automatiquement la validit\'e du travail.}

\medskip
    
{\large Map maker, \href{https://github.com/ThomasRanvier/map_maker}{(github)}}
\\
{\small Projet effectu\'e lors du semestre Erasmus en Su\`ede. 
L'objectif \'etait de d\'evelopper un algorithme permettant \`a un robot d'explorer son environnement en construisant une carte du monde des zones parcourues.
Le robot est simul\'e gr\^ace au logiciel Microsoft Robotic Developer Studio.
La cr\'eation de la carte est effectu\'ee gr\^ace \`a l'analyse des capteurs laser du robot.}
%Le chemin \`a suivre par le robot est d\'etermin\'e de mani\`ere \`a explorer les zones non connues.}

\medskip
    
{\large Line following robot, \href{https://github.com/ThomasRanvier/line_following_robot}{(github)}}
\\
{\small Projet effectu\'e lors du semestre Erasmus en Su\`ede. 
L'objectif \'etait de cr\'eer un robot capable de suivre une ligne gr\^ace \`a des capteurs infra-rouges.
Le robot utilise un BeagleBone Black et du mat\'eriel \'electronique de base.}

\medskip

{\large Cocke-Younger-Kasami Parser analysis, \href{https://github.com/ThomasRanvier/cyk_algorithm_analysis}{(github)}}
\\
{\small Projet effectu\'e lors du semestre Erasmus en Su\`ede. 
L'objectif \'etait d'analyser les performances de l'algorithme CYK en comparant des impl\'ementations utilisants diff\'erents paradigmes de d\'eveloppement (Divide \& conquer, Top-down, Bottom-up).
Puis de d\'evelopper un algorithme permettant de parser des cha\^ines de caract\`eres \`a partir d'une grammaire lin\'eaire et un second permettant de proposer des corrections \`a la chaine de caract\`ere pars\'ee.}

\medskip

{\large Faces recognition, \href{https://github.com/ThomasRanvier/faces_recognition_nn}{(github)}}
\\
{\small Projet effectu\'e lors du semestre Erasmus en Su\`ede.
L'objectif \'etait de d\'evelopper un single layer perceptron permettant la classification d'images de t\^etes représentants diff\`erentes \'emotions.}

\medskip

{\large SIGMA}
\\
{\small D\'eveloppement d'un nouvel ERP sous WinDev pour la compagnie Disa.
Utilis\'e quotidiennement par une centaine d'employ\'es, il propose des applications adapt\'ees aux besoins sp\'ecifiques des employ\'es de Disa.}

\medskip

%{\large MiLiCEM}
%\\
%{\small D\'eveloppement d'une application web (PHP) au sein d'Orange, permettant la mettant la vérification à distance du bon d\'eroulement d'op\'erations de migration de liens effectu\'ees sur le terrain.
%Application utilis\'ee quotidiennement par les membres de mon service (D\'etection Analyse Orientation) et d\'evelopp\'ee \`a \'echelle nationale au sein d'Orange apr\`es mon d\'epart.}

%\medskip
